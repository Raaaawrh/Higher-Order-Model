\section*{Уранение теплопроводности}
Уравнение теплопроводности имеет вид
\begin{equation*}
    \rho
    c_p
    \frac
        {\partial \Theta}
        {\partial t}
    =
    k_i
    \frac
        {\partial^2 \Theta}
        {\partial z^2}
    -
    \rho
    c_p
    \left(
        u
        \frac
            {\partial \Theta}
            {\partial x}
        +
        v
        \frac
            {\partial \Theta}
            {\partial y}
        +
        w
        \frac
            {\partial \Theta}
            {\partial z}
    \right)
    +
    2
    \dot{\varepsilon}
    \sigma
\end{equation*}

Это уравнение имеет распространные в моделировании льда упрощения:
\begin{itemize}
    \item лед считается несжимаемой средой
    \item лед считается изотропным (т.е. его теплофизические
    характеристики ($c_p, k_i$ и др.) всюду постоянны)
    \item температурное поле представляется в виде изотермических 
    плоскостей с горизонтальной ориентацией. То есть градиент
    температурного поля направлен всегда вдоль оси z (его составляющие
    вдоль x и y осей пренебрежимо малы)
\end{itemize}

После замены координат уравнение принимает следующий вид.
\begin{equation*}
    \frac   
        {\partial \Theta}
        {\partial t'}
    +
    \frac   
        {\partial \Theta}
        {\partial \xi}
    \left(
        a_x
        u
        +
        a_y
        v
        +
        a_z
        \left(
            \frac   
                {\partial s}
                {\partial t'}
            -
            \xi
            \frac   
                {\partial H}
                {\partial t'}
            -
            w
        \right)
    \right)
    +
    u
    \frac   
        {\partial \Theta}
        {\partial x'}
    +
    v
    \frac   
        {\partial \Theta}
        {\partial y'}
    -
    \frac
        {k}
        {H^2 \rho c_p}
    \frac   
        {\partial^2 \Theta}
        {\partial \xi^2}
    =
    \frac
        {2}
        {\rho c_p}
    \dot{\varepsilon}
    \sigma
\end{equation*}

Дискретизация также проводится с использованием центральных
разностей во внутренних узлах сетки и с сиспользвоанием трехточечных
односторонних разностей на границе домена.
