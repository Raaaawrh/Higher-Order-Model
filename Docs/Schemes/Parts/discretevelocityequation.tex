\section*{Дискретизация уравнений компонентов скорости}

Проведем дискретизацию уравнений для компонентов $u$ и $v$
скорости среды. Вывод проведем только с уравнением для
компоненты $u$, так как уравнения идентичны с учетом замен
$u \leftrightarrow v$ и $x \leftrightarrow y$, и получить
второе можно заменой указанных переменных и индексов
дискретизации.

Исходное уравнение
\begin{equation*}
    \begin{split}
        4
        \frac
        {\partial }
        {\partial x'}
        \left(
        \eta
        \frac
        {\partial u}
        {\partial x'}
        \right)
        &
        +
        4
        a_x
        \frac
        {\partial }
        {\partial x'}
        \left(
        \eta
        \frac
        {\partial u}
        {\partial \xi}
        \right)
        +
        \frac
        {\partial }
        {\partial y'}
        \left(
        \eta
        \frac
        {\partial u}
        {\partial y'}
        \right)
        +
        a_y
        \frac
        {\partial }
        {\partial y'}
        \left(
        \eta
        \frac
        {\partial u}
        {\partial \xi}
        \right)
        +
        4
        a_x
        \frac
        {\partial }
        {\partial \xi}
        \left(
        \eta
        \frac
        {\partial u}
        {\partial x'}
        \right)
        +
        a_y
        \frac
        {\partial }
        {\partial \xi}
        \left(
        \eta
        \frac
        {\partial u}
        {\partial y'}
        \right)
        +
        \\
        &
        +
        \left(
        4
        a^2_x
        +
        a^2_y
        +
        a^2_z
        \right)
        \frac
        {\partial }
        {\partial \xi}
        \left(
        \eta
        \frac
        {\partial u}
        {\partial \xi}
        \right)
        +
        \left(
        4
        b_x
        +
        b_y
        \right)
        \left(
        \eta
        \frac
        {\partial u}
        {\partial \xi}
        \right)
        =
        \rho
        g
        \frac
        {\partial s}
        {\partial x'}
        -
        \\
        &
        -
        2
        \frac
        {\partial }
        {\partial x'}
        \left(
        \eta
        \frac
        {\partial v}
        {\partial y'}
        \right)
        -
        2
        a_y
        \frac
        {\partial }
        {\partial x'}
        \left(
        \eta
        \frac
        {\partial v}
        {\partial \xi}
        \right)
        -
        \frac
        {\partial }
        {\partial y'}
        \left(
        \eta
        \frac
        {\partial v}
        {\partial x'}
        \right)
        -
        a_x
        \frac
        {\partial }
        {\partial y'}
        \left(
        \eta
        \frac
        {\partial v}
        {\partial \xi}
        \right)
        -
        \\
        &
        -
        a_y
        \frac
        {\partial }
        {\partial \xi}
        \left(
        \eta
        \frac
        {\partial v}
        {\partial x'}
        \right)
        -
        2
        a_x
        \frac
        {\partial }
        {\partial \xi}
        \left(
        \eta
        \frac
        {\partial v}
        {\partial y'}
        \right)
        -
        3
        a_x
        a_y
        \frac
        {\partial }
        {\partial \xi}
        \left(
        \eta
        \frac
        {\partial v}
        {\partial \xi}
        \right)
        -
        3
        c_{xy}
        \left(
        \eta
        \frac
        {\partial v}
        {\partial \xi}
        \right)
    \end{split}
\end{equation*}

Дискретизацию уравнения можно получить, просто подставив указанные
выше конечно-разностные аппроксимации операторов. В таком случае
получим систему линейных уравнений. Однако в совокупности уравнения
для $u$ и $v$ компонент образуют нелинейную задачу, которую решать
предлагается (J.J. Furst. Improved convergence and stability properties in a
three-dimensional higher-order ice sheet model, 2011) итерациями
Пикара. Опишем кратко.
Уравнения для $u$ и $v$ образуют две системы
\begin{equation*}
    \begin{cases}
        A_u(v^{r-1}) u^{r} = b_{u}(v^{r-1})
        \\
        A_v(u^{r-1}) v^{r} = b_{v}(u^{r-1})
    \end{cases}
\end{equation*}
Пусть есть некоторое начальное приближение $(u^0, v^0)$. Используя
это приближение вычисляется приближение вязкости
\begin{equation*}
    \eta
    =
    \onehalf
    A^{\frac{-1}{n}}(\Theta^{*})
    \left\{
    \left(
    \frac
    {\partial u}
    {\partial x}
    \right)^2
    +
    \left(
    \frac
    {\partial v}
    {\partial y}
    \right)^2
    +
    \frac
    {\partial u}
    {\partial x}
    \frac
    {\partial v}
    {\partial y}
    +
    \frac
    {1}
    {4}
    \left(
    \frac
    {\partial u}
    {\partial y}
    +
    \frac
    {\partial v}
    {\partial x}
    \right)^2
    +
    \frac
    {1}
    {4}
    \left(
    \frac
    {\partial u}
    {\partial z}
    \right)^2
    +
    \frac
    {1}
    {4}
    \left(
    \frac
    {\partial v}
    {\partial z}
    \right)^2
    +
    \dot{\varepsilon^2_0}
    \right\}^{\frac{1-n}{2n}}
\end{equation*}

или, после преобразования координат:

\begin{equation*}
    \begin{split}
        \eta
        =
        &
        \onehalf
        A^{\frac{-1}{n}}(\Theta^{*})
        \left\{
        \left(
        \frac
        {\partial u}
        {\partial x^{'}}
        +
        a_x
        \frac
        {\partial u}
        {\partial \xi}
        \right)^2
        +
        \left(
        \frac
        {\partial v}
        {\partial y^{'}}
        +
        a_y
        \frac
        {\partial v}
        {\partial \xi}
        \right)^2
        +
        \left(
        \frac
        {\partial u}
        {\partial x^{'}}
        +
        a_x
        \frac
        {\partial u}
        {\partial \xi}
        \right)
        \left(
        \frac
        {\partial v}
        {\partial y^{'}}
        +
        a_y
        \frac
        {\partial v}
        {\partial \xi}
        \right)
        +
        \right.
        \\
        &
        \left.
        +
        \frac
        {1}
        {4}
        \left(
        \frac
        {\partial u}
        {\partial y^{'}}
        +
        a_y
        \frac
        {\partial u}
        {\partial \xi}
        +
        \frac
        {\partial v}
        {\partial x^{'}}
        +
        a_y
        \frac
        {\partial v}
        {\partial \xi}
        \right)^2
        +
        \frac
        {1}
        {4}
        \left(
        a_z
        \frac
        {\partial u}
        {\partial \xi}
        \right)^2
        +
        \frac
        {1}
        {4}
        \left(
        a_z
        \frac
        {\partial v}
        {\partial \xi}
        \right)^2
        +
        \dot{\varepsilon^2_0}
        \right\}^{\frac{1-n}{2n}}
    \end{split}
\end{equation*}

где
\begin{equation*}
    A (\Theta^*)
    =
    m
    \left(
        \frac
            {1}
            {B_0}
    \right)^n
    \exp
    \left(
        \frac
            {3C}
            {(\Theta_r - \Theta^*)^K}
        -
        \frac
            {Q}
            {R\Theta^*}
    \right)
\end{equation*}

Более подробное описание можно найти в документации CISM 2.0 либо
в статье (F. Pattyn A new three-dimensional higher-order 
thermomechanical ice sheet model: Basic sensivity, ice stream
development, and ice flow across subglacial lakes, 2003)

После производим решение первой системы для $u^1$, используя
$(u^0, v^0)$ и решение $v^1$. Потом повторяем с использованием
$(u^1, v^1)$ и повторяем до сходимости.

Другой подход заключается в одновременном решении совокупности
систем для $u$ и $v$. Тогда размер системы увеличится в 2 раза, и
вычислительная сложность примено в 4 раза. J.J. Furst утверждает,
что вполне достаточно исполььзовать раздельное решение, а совместное
решение ничего кроме увеличения вычислительных затрат не влечет.

При решении необходимо установить граничные условия. На границах домена
по осям $x$ и $y$ принято ставить условие нулевой скорости.
На поверхности применяется так называемое граничное условие свободной
поверхности (stress-free surface boundary condition)

Напишем его сразу в преобразованных координатах
\begin{equation*}
    4
    \frac
    {\partial s}
    {\partial x'}
    \frac
    {\partial u}
    {\partial x'}
    +
    \frac
    {\partial s}
    {\partial y'}
    \frac
    {\partial u}
    {\partial y'}
    +
    \left(
    4
    a_x
    \frac
    {\partial s}
    {\partial x'}
    +
    a_y
    \frac
    {\partial s}
    {\partial y}
    +
    a_z
    \right)
    \frac
    {\partial u}
    {\partial \xi}
    =
    -
    \frac
    {\partial s}
    {\partial y'}
    \frac
    {\partial v}
    {\partial x'}
    -
    2
    \frac
    {\partial s}
    {\partial x}
    \frac
    {\partial v}
    {\partial y'}
    -
    \left(
    2
    a_y
    \frac
    {\partial s}
    {\partial x'}
    +
    a_x
    \frac
    {\partial s}
    {\partial y}
    \right)
    \frac
    {\partial v}
    {\partial \xi}
\end{equation*}

Для вычисления всех производных желательно использовать аппроксимации
второго порядка. Во внутренних узлах -- центральные разности. На
границах использовать односторонние разности (upstream finite-difference
approximations), вывод которых несложен и проведен в (F. Pattyn Transient glacier response with a higher-order numerical
ice-flow model, 2002). Любые уточнения можно найти также в документации
к проекту CISM 2.0.

Компоненту w после итераций получаем из условия несжимаемости
\begin{equation*}
    w(z)
    -
    w(b)
    =
    -
    \int_{b}^{z}
    \left(
    \frac
    {\partial u}
    {\partial x}
    +
    \frac
    {\partial v}
    {\partial y}
    dz
    \right)
\end{equation*}

Простым численным интегрированием с использвоанием квадратурных
формул.
